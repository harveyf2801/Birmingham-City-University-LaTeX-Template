\section{Evaluation}
    \note{
        This section is the first section of the assessment that is completely new to the report. 
        
        The evaluation section should provide testing of the artefact and overall project. This will express ideas in answer any research question. Depending on the evaluation chosen, a variety of possible layouts may result. Nonetheless, it is good practice to consider the evaluation section to be divided into two subsections based on the experimental design and the outcomes.
    }
    
    \subsection{Evaluation Methodology}
        \note{
            Evaluation/Experimental methodology: Here you describe the selected approach to evaluating your design, as well as the motivation for the approach. If this is a standard way of measuring particular phenomena, then it can be motivated through citation. The experimental design of your evaluation will include various subsections possibly including:
            
            NB: The following sub-subsections (i.e., 4.1.1 through 4.1.3) may not be relevant to your specific project topics, so you should discuss the sections with your supervisor to tailor this to your needs.
        }
        
        \subsubsection{Evaluation Metrics}
            \note{
                The specific metrics being used to assess success.
            }
            
        \subsubsection{Baseline Systems}
            \note{
                Systems under analysis or Baseline systems: The designs being tested apart from the one proposed in the method section. Note that these may also be variants of the proposed approach.
            }
            
        \subsubsection{Dataset}
            \note{
                A collection of data that is used to provide reliable consistency in comparative assessments across systems. Depending on your chosen project this may or may not be relevant.

                The evaluation outcomes will include various subsections including:
            }
        
    \subsection{Results}
        \note{
            This section is mandatory. Here you will describe the detailed measurements of your system. Which trends appear? Which design performed best across which evaluations? If you have tables or figures that show the performance of your design (and possibly others) refer to these in the text as you explain the output. You may also wish to provide exemplar outputs of the design, which demonstrate the performance of your system, alongside a discussion of the result in the text.
        }
    
    \subsection{Discussion}
        \note{
            This is a crucial section of the report and should be explored in great depth. The results from the previous subsection are here explained with consideration to the context of the project. This is the area in which you can confirm similarity or difference between trends that appear in your research with that of others that you have discussed in your literature review. You may also hypothesize why you believe certain outputs/phenomena have occurred. This is a deeper analysis in which you piece apart the results to determine the underlying causes of the recorded output.
            
        For business and management related projects, the presentation of findings may be integrated within discussion sections. Limitations of the chosen methods should be identified and ways to overcome them suggested. If compromises have to be accepted, for example in time and cost. Such limitations and problems should be identified together with how they are to be overcome and/or the compromises that will have had to be made.

        Depending on the nature of the project, and particularly with certain business topics for which the main outcomes are recommendations on various management related aspects, the results and discussion chapters may be integrated within chapter(s) of findings covering the relevant project objectives. In this case this chapter could be entitled Recommendations.
        }