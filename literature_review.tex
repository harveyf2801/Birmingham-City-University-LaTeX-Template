\section{Literature Review}
    \note{This is an organised, critical report detailing the various sources of applicable research around relevant topics. Note that this is an indicative example of the report structure that should be used. Refer to the Project Handbook (p. 18-20) and video tutorials in weeks 4-7 for an explanation of content to be included. This should be discussed with your supervisor well in advance, as subject areas may have different approaches that are not in line with the one presented here. The literature review should be approximately 2000 words.}
	
    \subsection{Themes}
        \note{Discuss what areas (i.e., themes) need to be explored and why. Typically, there will be around 5 or 6 themes required. You may refer to a mind map in the Appendix showing themes if it helps (but this is not necessary). State the keywords used, which are associated with each theme. Example phrasing: 'A thematic approach has been undertaken to identify the areas that need to be understood to develop the artefact. From this, a number of keywords for each have been used to obtain information from the literature'. List your themes and give example keywords. Note that one of the themes to consider in the literature review is how other researchers approach the topic of evaluation.}
        
    \subsection{Review of Literature}
        \note{This subsection will comprise the main body of the literature review. It will contain a historical overview of the literature relevant to each theme in your project. Relational information about each reference will be presented to provide context for various sources (i.e., brief description of important aspect(s) of each source) and a system of categorisation of topics (i.e., modes of interpreting sources) will be used to separate sources into different classes. This can either be written as one subsection for each theme or as two subsections (i.e., Review and Theory) for each theme as below. If one subsection is used, the information in the Review and Theory sections below must be woven together for each theme discussed. This means you will have one subsection for each theme. This will be determined through discussion with your supervisor.}

    \subsubsection{Review}
        \note{Should be who did what and why for each theme. While this is a critique, it is NOT your opinion on research undertaken by other researchers. It must include many citations for each theme (at least 8 references for each theme) and a useful ordering of information is based a timeline in years. This is NOT a description of a paper or article in a list format.}

    \subsubsection{Theory}
        \note{Details of how things work. This is very different from the Review, which provides cursory links between research. It should include design methods (e.g., equations, algorithms) that will be used and so on. There is no need to start from basics (e.g., V=IR, syntax error) as the reader will have some knowledge.}

    \subsection{Summary}
        \note{Provide an overview of the topics discussed and information presented. Given this information that you have read, what conclusions may be drawn from this? How does this lead into the choices made going into the following section?}