\section{Project Design and Methods}
	\note{The design and methods section explains the methods, techniques and overall design to be implemented in the project artefact and reflects the work that you have undertaken since doing the research presented in the literature review (but not in a timeline view). The design and methods section is informed by the literature review in that it is a product of the knowledge gained through the research undertaken to understand what is crucial to the project. Please refer to the Project Handbook (p. 20-22) and video tutorial in Week 9-10 for further information. The design and methods section should be approximately 2000 words.}
    
	\subsection{Introduction}
		\note{The introduction should also introduce the reader to how you have structured this section. Many of the below subsections may be combined and presented together in fewer subsections (e.g., combining Limitations and Options, Design Specification, User Requirements, and Concept Solution into a larger all-encompassing section). Again, this will be determined through discussions with your supervisor.}
	
	\subsection{Methodology}
		\note{There are many specialist methodologies that exist to conduct projects. Identify the type of methodology used and state why (e.g., Waterfall). This only needs to be a paragraph but shows you understand the approach taken. A flow diagram may be useful to show your methods. Note that this is the way in which you develop something, not the component by component creation of the artefact.}
	
	\subsection{Limitations and Options}
		\note{A useful way to obtain your design specification is to consider the methods discussed by the various authors in the literature review. You can then do a comparison of these and identify the best option for you for a particular theme (e.g., based on cost, availability). Provide a description or tables to indicate limitations and options for each theme to be considered.}
	
	\subsection{Design Specification/User Requirements}
		\note{From your limitations and options section, you can now detail the specification or user requirements (i.e., a list). If this is a research-based project you can now identify the method of obtaining primary results from your chosen testing strategies.}
	
	\subsection{Concept Solution}
		\note{Having obtained a specification to design against, you now need to produce a solution. Usually there are several ways that you can approach this so discuss this and decide on the final version. At this stage you should be able to define a block diagram of what it is you will be producing showing each stage that has to be considered.}
	
	\subsection{Testing Strategies}
		\note{Having obtained a specification to design against, you now need to produce a solution. Usually there are several ways that you can approach this so discuss this and decide on the final version. At this stage you should be able to define a block diagram of what it is you will be producing showing each stage that has to be considered.}
	
	\subsection{Design and Development}
		\note{Details from your block/flow diagram are used in explaining the design of your artefact. You will need to make use of equations/algorithms/CAD where appropriate. Reproducibility is key here; assume that by the end of this section you can give the details to someone else and they can produce your artefact based on the information provided.}
	
	\subsection{Testing}
		\note{If you have not detailed the testing earlier then now is the time to do it. Consider how many different tests you will do. You cannot do everything so discuss this with your supervisor. Assume that this is like any science testing you were taught at school/college which followed the list of apparatus/method and so on. Obviously here you are describing what is to be done but it shows you have thought it through and know what resources will be needed. Detail each test as a sub-subsection.}
	
	\subsection{Summary and Conclusions}
		\note{Give a summary of the main points from the design and methods section and explain the next steps. This does not need to be long.}