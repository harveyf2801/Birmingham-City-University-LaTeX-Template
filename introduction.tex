\section{Introduction}
    \note{
        This will clearly state the rationale and objectives of the research and contain much of the same information present in the proposal (e.g., problem definition, scope, rationale, aims and objectives). Begin with a brief introduction to provide preparation for the rest of the report, with a clear outline of what was done and the rationale for the work. Much of the information that you have already written will be utilised throughout this section, however it should be specifically tailored to this assessment point.
    
        Start the introduction by answering the question: What is the subject of the project?
    }

    \subsection{Problem Definition}
        \note{
            A statement of the problem, with its significance and origin. If applicable, make reference to the company or industry that led to the project definition. 
        }
        
    \subsection{Scope}
        \note{
            This section identifies the boundaries of the project, what was included and what was excluded from the final project. This should be justified and underpinned by research. 
        }
    
    \subsection{Rationale}
        \note{
            Why has the topic been chosen? This may be because of lack of research in the area, to shed more ideas and opinion, in response to a request, (e.g., from a company, organisation or relevant current issue). What benefits can be identified from completing the project? This should be more than personal interest—you should be able to identify a company, organisation or other defined group that will benefit from the work.
        }
    
    \subsection{Project Aim and Objectives}
        \note{
            There should be a brief and precise statement of overall aim—what is intended to be attained? There should follow a list, using bullet points, of objectives—the completion of which will lead to the attainment of the aim. The objectives are developed from the aim and can be viewed as incremental stages in the attainment of the aim(s). Bloom’s Taxonomy is useful in writing these objectives (see Moodle site).
        }
    
    \subsection{Background Information}
        \note{
            A further section of background information will depend on the topic area of the project, but could include hypotheses and theory, which are to be tested in the course of undertaking the project. This is an optional subsection but may be useful in defining the contextual information.
        }