\section{Introduction}
    \note{
        This will clearly state the rationale and objectives of the research and contain much of the same information present in the proposal (e.g., problem definition, scope, rationale, aims and objectives). Begin with a brief introduction to provide preparation for the rest of the report, with a clear outline of what was done and the rationale for the work. Much of the information that you have already written will be utilised throughout this section, however it should be specifically tailored to this assessment point.
    
        Start the introduction by answering the question: What is the subject of the project?
    }
    
    \subsection{Aims}
        \note{
            There should be a brief and precise statement of overall aim—what is intended to be attained?
        }
    
    \subsection{Objectives}
        \note{
            There should follow a list, using bullet points, of objectives—the completion of which will lead to the attainment of the aim. The objectives are developed from the aim and can be viewed as incremental stages in the attainment of the aim(s). Bloom’s Taxonomy is useful in writing these objectives (see Moodle site).
        }
        \begin{itemize}
            \item One
            \item Two
            \item Three
            \item Four
            \item Five
        \end{itemize}